\documentclass{article}
\usepackage{fontspec}
\usepackage{xunicode}
\usepackage{polyglossia}
\usepackage{hyperref}

\setmainlanguage{english}
\setotherlanguages{telugu}
\newfontfamily\telugufont[Script=Telugu]{Noto Serif Telugu}

\title{Telugu Document Template in \LaTeX}
\author{vkirank (\url{vkirank.com})}
\date{July 30, 2021}

\begin{document}
\maketitle
\section{Introduction}
This is a sample document containing both Telugu and English text.
Getting the Telugu language to work with \LaTeX is difficult, so I've uploaded this sample document (along with its source code) onto Github.
Feel free to use it as a template for whatever you want.
\section{Compiling Telugu Documents}
To get Telugu to work with \LaTeX, you need to compile your .tex documents using XeLaTex.
\textbf{Do NOT compile this file using pdflatex!!!} It won't work properly.
To compile this .tex file, you should run the following command:
\begin{verbatim}
xelatex telugutest.tex
\end{verbatim}
\section{Sample Telugu Text}
\begin{telugu}
ఆదియందు దేవుడు భూమ్యాకాశములను సృజించెను.
భూమి నిరాకారముగాను శూన్యముగాను ఉండెను; చీకటి అగాధ జలము పైన కమ్మియుండెను; దేవుని ఆత్మ జలములపైన అల్లాడుచుండెను.
(\url{https://www.sajeevavahini.com/telugubible/})
\linebreak
౦ ౧ ౨ ౩ ౪ ౫ ౬ ౭ ౮ ౯
\end{telugu}
\end{document}
